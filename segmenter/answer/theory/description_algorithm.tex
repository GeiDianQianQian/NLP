%=======================CIS 9615 LaTeX template==================
%
% How to use:
%    1. Update your information in section "A" below
%    2. Write your answers in section "B" below. Precede answers for all 
%       parts of a question with the command "\question{n}{desc}" where n is
%       the question number and "desc" is a short, one-line description of 
%       the problem. There is no need to restate the problem.
%    3. If a question has multiple parts, precede the answer to part x with the
%       command "\part{x}".
%    4. If a problem asks you to design an algorithm, use the commands
%       \algorithm, \correctness, \runtime to precede your discussion of the 
%       description of the algorithm, its correctness, and its running time, respectively.
%    5. You can include graphics by using the command \includegraphics{FILENAME}
%
\documentclass[11pt]{article}
\usepackage{mathtools}
\usepackage{graphicx}
\usepackage{subfigure}
\usepackage{amssymb}
\usepackage{multirow}
\usepackage{algorithm}
\usepackage{url}
\usepackage[noend]{algorithmic}
\usepackage[numbers,sort&compress]{natbib}
\usepackage{color}
\usepackage{colortbl}
\usepackage{multirow}
\usepackage{hhline}
\usepackage{textcomp}
\setlength{\arrayrulewidth}{1pt}

\hyphenation{op-tical net-works semi-conduc-tor}

\newtheorem{theorem}{\textbf{Theorem}}
\newtheorem{lemma}{\textbf{Lemma}}
\newtheorem{proposition}{\textbf{Proposition}}
\newtheorem{definition}{\textbf{Definition}}
\newtheorem{corollary}{\textbf{Corollary}}
\newtheorem{conjecture}{\textbf{Conjecture}}
\usepackage{amsmath,amssymb,amsthm}
\usepackage{graphicx}
\usepackage{subfigure}
\usepackage[margin=1in]{geometry}
\usepackage{fancyhdr}
\usepackage{algorithm}
\usepackage{url}
\usepackage[noend]{algorithmic}
\setlength{\parindent}{0pt}
\setlength{\parskip}{5pt plus 1pt}
\setlength{\headheight}{13.6pt}
\newcommand\question[2]{\vspace{.25in}\hrule\textbf{#1: #2}\vspace{.5em}\hrule\vspace{.10in}}
\renewcommand\part[1]{\vspace{.10in}\textbf{(#1)}}
%\newcommand\algorithm{\vspace{.10in}\textbf{Algorithm: }}
\newcommand\prof{\vspace{.10in}\textbf{Proof: }}
\newcommand\correctness{\vspace{.10in}\textbf{Correctness: }}
\newcommand\base {\vspace{.10in}\textbf{Base Case: }}
\newcommand\inductive {\vspace{.10in}\textbf{Inductive Step: }}
\newcommand\runtime{\vspace{.10in}\textbf{Running time: }}
\pagestyle{fancyplain}
\lhead{\textbf{\NAME}}
\lhead{\textbf{\NAME\ (\ID)}}
\chead{\textbf{\HWNUM}}
\rhead{CMPT 825, Fall 2017}
%\begin{document}\raggedright
\begin{document}
%Section A==============Change the values below to match your information==================
\newcommand\NAME{Turash and Luiz}  % your name
\newcommand\ID{tmosharr \& lperesde}     % your Temple AccessNet Account
\newcommand\HWNUM{}              % the homework number



%Section B==============Put your answers to the questions below here=======================

% no need to restate the problem --- the graders know which problem is which,
% but replacing "The First Problem" with a short phrase will help you remember
% which problem this is when you read over your homeworks to study.
%send to : HW3Subm.43vjts2a9ie04m7x@u.box.com
\question{A}{Theory} 
We consider a unigram as a single word and a bigram as an ordered pair of words in the form $\{w_1,w_2\}$ separated by space.  We have two dictionaries $Pw$ and $Pw2$ that contains unigrams and bigrams respectively along with their frequencies. We use the following notations. 
\begin{table}[h]
	\vspace{-4pt}
	\centering
	\renewcommand\arraystretch{1.25}
	\begin{tabular}{|c|c|}
		\hline
		Notations   & Meaning  \\
		\hline
		Pw[$w$] & frequency of $w$ in Pw dictionary    \\
		\hline
		Pw2[$w_1,w_2$] & frequency of \{$w_1,w_2$\} bigram in Pw2 dictionary    \\
		\hline
		$V$ & Total number of different unigrams or bigrams  \\
		\hline
		$N$ & Sum of frequencies  \\
		\hline
		$voc[w]$ & Number of different bigrams started by $w$\\
		\hline
	\end{tabular}
	\vspace{-6pt}
\end{table}   



There can be different cases that might happen when we analyze any bigram. However, when we analyze the probability of the very first word of the document or when we can not find the first word $w_1$ of an ordered pair \{$w_1,w_2$\} in the dictionary Pw, we only use the unigram model to assign the probability of the word.

According to the unigram model, The probability of a word $w$ is given by:   
\begin{equation}
P(w)=\frac{Pw[w]+1}{Pw.N+Pw.V+1}
\end{equation}
if it is found in Pw or
\begin{equation}
P(w)=\frac{1}{Pw.N+Pw.V+1}
\end{equation}
if it is not found in Pw

However, we use bigram model for every ordered pair \{$w_1,w_2$\} when $w_1$ is found in Pw. According to the bigram model, the conditional probability of $w_2$ given that we already observed $w_1$ is calculated. There can be any of the following three cases. 

\part{i} Bigram \{$w_1,w_2$\} in Pw2\\
In this case there is a connection between two known words. We consider conditional probability for $w_2$ as:
\begin{equation}
	P(w_2|w_1)=\frac{Pw2[w_1,w_2]+1}{Pw[w_1]+Pw2.voc[w_1]+1}
\end{equation}

\part{ii} Bigram not in Pw2, $w_2$ in Pw\\
Although there is no observed connection, both are known words. In this case, we assign the conditional probability based on the frequency of $w_2$ in Pw. 
\begin{equation}
	P(w_2|w_1)=\frac{Pw[w_2]+1}{(Pw[w_1]+Pw2.voc[w_1]+1)(Pw.N+Pw.V+1)}
\end{equation}

\part{iii} Bigram not in Pw2, $w_2$ is not in Pw\\
In this case, $w_2$ is completely unknown. Instead of assigning a zero probability, we assign probability of $w_2$ as \\
\begin{equation}
P(w_2|w_1)=\frac{1}{(Pw[w_1]+Pw2.voc[w_1]+1)(Pw.N+Pw.V+1)}
\end{equation}

\question{B}{Data Structure}

\begin{table}[h]
	\vspace{-4pt}
	\centering
	\renewcommand\arraystretch{1.25}
	\begin{tabular}{|c|c|}
		\hline
		Data Structure   & Description  \\
		\hline
		Pw,Pw2 & Dictionaries with frequencies of unigrams and bigrams respectively \\
		\hline
		chart &Dynamic programming Table to store the words\\
		\hline
		entry & A structure \{word, start position, log probability, back pointer\}  \\
		\hline
		pq & priority queue of entries, priority= lowest starting   \\
		\hline
	\end{tabular}
	\vspace{-6pt}
\end{table}

\question{C}{Algorithm} 

The algorithm is given below:		

\begin{algorithm}[htb]
	\renewcommand{\algorithmicrequire}{\textbf{Input:}\hspace{10.7pt}}
	\renewcommand\algorithmicensure {\textbf{Output:} }
	\caption{Word Segmentation}
	\label{alg:pbg}
	\begin{algorithmic}[1]
		\REQUIRE File with non segmented words $Input$,\\
		unigram dictionary $Pw$,\\ bigram dictionary $Pw_2$\\ 
		\ENSURE File with segmented words \\
		\vspace{3pt}
		\STATE Initialize $chart=\{\}$ and $maxlen \gets $ longest bigram length
		\STATE Initialize priority queue with starting position as priority
		\FOR{$l$ in  range(1,...,maxlen)}
		\STATE $w_1 \gets$ substring input[0,...,l-1]
		\IF {$w_1$ is found in Pw}
		\STATE calculate $P(w_1)$ according to unigram model 
		\STATE add $\{w_1,0,P[w_1],None\}$ into pq
		\ENDIF
		\ENDFOR
		\IF {nothing inserted in last for loop}
		\STATE $w_1 \gets$ input[0]
		\STATE calculate $P(w_1)$ according to unigram model  
		\STATE add $\{w_1,0,Pw[w_1],None\}$ into pq
		\ENDIF
		\WHILE{pq not empty}
		\STATE $item \gets$ highest priority item from pq
		\STATE $end \gets item.length+item.start$ 
		\IF {$item.probability > chart[end].probability$}
		\STATE $chart[end] \gets item$
		\ENDIF
		
		\STATE $nextstart \gets end+1$
		\IF {$w_1$ is found in Pw}
		
		\FOR{$l$ in  range(1,...,maxlen)} 
		\STATE $w_2 \gets$ substring input[nextstart,...,nextstart+l-1] 
		\STATE calculate $P(w_2|w_1)$ according to bigram model 
		\STATE add $\{w_2,nextstart, P(w_1)*P(w_2|w_1),item\}$ into pq
		\ENDFOR
		
		\IF {nothing inserted in last for loop}
		\STATE $w_1 \gets$ input[0]
		\STATE calculate $P(w_1)$ according to unigram model  
		\STATE add $\{w_2,nextstart,P(w_1)*P(w_2|w_1),item\}$ into pq
		\ENDIF
		
		\ELSE
		\FOR{$l$ in  range(1,...,maxlen)} 
		\STATE $w_2 \gets$ substring input[nextstart,...,nextstart+l-1] 
		\STATE calculate $P(w_2)$ according to unigram model 
		\STATE add $\{w_2,nextstart,P(w_1)*P(w_2),item\}$ into pq
		\ENDFOR
		
		\IF {nothing inserted in last for loop}
		\STATE $w_2 \gets$ input[nextstart]
		\STATE calculate $P(w_2)$ according to unigram model  
		\STATE add $\{w_2,nextstart,P(w_1)*P(w_2),item\}$ into pq
		\ENDIF
		\ENDIF
		\ENDWHILE
		\STATE $item \gets$ chart[end]
		\WHILE{$item is not None$}
		\STATE add item.word to wordlist
		\STATE $item \gets item.backpointer$
		\ENDWHILE
		\STATE print wordlist in reverse order
	\end{algorithmic}
\end{algorithm}





\end{document}
