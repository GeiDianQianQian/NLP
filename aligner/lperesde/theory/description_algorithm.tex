%=======================CIS 9615 LaTeX template==================
%
% How to use:
%    1. Update your information in section "A" below
%    2. Write your answers in section "B" below. Precede answers for all 
%       parts of a question with the command "\question{n}{desc}" where n is
%       the question number and "desc" is a short, one-line description of 
%       the problem. There is no need to restate the problem.
%    3. If a question has multiple parts, precede the answer to part x with the
%       command "\part{x}".
%    4. If a problem asks you to design an algorithm, use the commands
%       \algorithm, \correctness, \runtime to precede your discussion of the 
%       description of the algorithm, its correctness, and its running time, respectively.
%    5. You can include graphics by using the command \includegraphics{FILENAME}
%
\documentclass[11pt]{article}
\usepackage{mathtools}
\usepackage{graphicx}
\usepackage{subfigure}
\usepackage{amssymb}
\usepackage{multirow}
\usepackage{algorithm}
\usepackage{url}
\usepackage[noend]{algorithmic}
\usepackage[numbers,sort&compress]{natbib}
\usepackage{color}
\usepackage{colortbl}
\usepackage{multirow}
\usepackage{hhline}
\usepackage{textcomp}
\setlength{\arrayrulewidth}{1pt}

\hyphenation{op-tical net-works semi-conduc-tor}

\newtheorem{theorem}{\textbf{Theorem}}
\newtheorem{lemma}{\textbf{Lemma}}
\newtheorem{proposition}{\textbf{Proposition}}
\newtheorem{definition}{\textbf{Definition}}
\newtheorem{corollary}{\textbf{Corollary}}
\newtheorem{conjecture}{\textbf{Conjecture}}
\usepackage{amsmath,amssymb,amsthm}
\usepackage{graphicx}
\usepackage{subfigure}
\usepackage[margin=1in]{geometry}
\usepackage{fancyhdr}
\usepackage{algorithm}
\usepackage{url}
\usepackage[noend]{algorithmic}
\setlength{\parindent}{0pt}
\setlength{\parskip}{5pt plus 1pt}
\setlength{\headheight}{13.6pt}
\newcommand\question[2]{\vspace{.25in}\hrule\textbf{#1: #2}\vspace{.5em}\hrule\vspace{.10in}}
\renewcommand\part[1]{\vspace{.10in}\textbf{(#1)}}
%\newcommand\algorithm{\vspace{.10in}\textbf{Algorithm: }}
\newcommand\prof{\vspace{.10in}\textbf{Proof: }}
\newcommand\correctness{\vspace{.10in}\textbf{Correctness: }}
\newcommand\base {\vspace{.10in}\textbf{Base Case: }}
\newcommand\inductive {\vspace{.10in}\textbf{Inductive Step: }}
\newcommand\runtime{\vspace{.10in}\textbf{Running time: }}
\pagestyle{fancyplain}
\lhead{\textbf{\NAME}}
\lhead{\textbf{\NAME\ (\ID)}}
\chead{\textbf{\HWNUM}}
\rhead{CMPT 825 and CMPT 413, Fall 2017}
%\begin{document}\raggedright
\begin{document}
%Section A==============Change the values below to match your information==================
\newcommand\NAME{Turash and Luiz}  % your name
\newcommand\ID{tmosharr \& lperesde}     % your Temple AccessNet Account
\newcommand\HWNUM{}              % the homework number



%Section B==============Put your answers to the questions below here=======================

% no need to restate the problem --- the graders know which problem is which,
% but replacing "The First Problem" with a short phrase will help you remember
% which problem this is when you read over your homeworks to study.
%send to : HW3Subm.43vjts2a9ie04m7x@u.box.com
\question{A}{Theory} 

The algorithm used on this assignment was the implementation of the baseline merged with the ability of finding null word alignments and the posterior probability, accounting both $p(f|e)$ and $p(e|f)$.

\question{B}{Algorithm} 

The algorithm for the word alignment is given below:	

\begin{algorithm}[htb]
	\renewcommand{\algorithmicrequire}{\textbf{Input:}\hspace{10.7pt}}
	\renewcommand\algorithmicensure {\textbf{Output:} }
	\caption{Word Alignment}
	\label{alg:pbg}
		\begin{algorithmic}[1]
		\vspace{3pt}
		\REQUIRE
		French/German sentences $f$ and English sentences $e$\\
		\ENSURE Aligned words \\
		\STATE Get count of English words $v_e$ and French/German words $v_f$
		\STATE Initialize dictionaries $t_1$, $t_2$, $count_e$, $count_f$  $count_{fe}$ and $count_{ef}$
		\REPEAT
		  \STATE Make NULL available
		  \FOR {each pair of sentences ($f$, $e$)}
			\FOR {each $f_i$ in $f$}
			  \STATE Initialize $c = 0$ and $z = 0$
			  \FOR {each $e_j$ in $e$}
			    \STATE normalize $z = 1.0 / v_f$ \textbf{if} first iteration \textbf{else} $z += t_1[(f_i,e_j)]$ 
			  \ENDFOR
			  \FOR {each $e_j$ in $e$}
			    \STATE normalize $c = (1.0/v_f)/z$ \textbf{if} first iteration \textbf{else} $c = t_1[(f_i, e_j)]/z$
			    \STATE increment both $count_e[e_j]$ and $count_{fe}[(f_i,e_j)]$ with the value of $c$
			  \ENDFOR
			\ENDFOR
			\FOR {each $e_i$ in $e$}
			  \STATE Initialize $c = 0$ and $z = 0$
			  \FOR {each $f_j$ in $f$}
			    \STATE normalize $z = 1.0 / v_e$ \textbf{if} first iteration \textbf{else} $z += t_2[(e_i,f_j)]$ 
			  \ENDFOR
			  \FOR {each $f_j$ in $f$}
			    \STATE normalize $c = (1.0/v_e)/z$ \textbf{if} first iteration \textbf{else} $c = t_2[(e_i, f_j)]/z$
			    \STATE increment both $count_f[f_j]$ and $count_{ef}[(e_i,f_j)]$ with the value of $c$
			  \ENDFOR
			\ENDFOR
		  \ENDFOR
		  \FOR {each pair $(e, f)$ in $count_{fe}$}
		    \STATE $t_1[(f,e)] = (count_{fe}[(f,e)] + 0.1)/(count_e[e] + 0.1 * v_f)$
		  \ENDFOR
		  \FOR {each pair $(e, f)$ in $count_{ef}$}
		    \STATE $t_2[(e,f)] = (count_{ef}[(e, f)] + 0.1)/(count_f[f] + 0.1 * v_e)$
		  \ENDFOR
		\UNTIL {\# of iterations not 5}
		\STATE Print the best word alignments according to the instructions of the baseline and ignore the alignments defined as \textit{NULL}
	\end{algorithmic}
\end{algorithm}

\end{document}