%=======================CIS 9615 LaTeX template==================
%
% How to use:
%    1. Update your information in section "A" below
%    2. Write your answers in section "B" below. Precede answers for all 
%       parts of a question with the command "\question{n}{desc}" where n is
%       the question number and "desc" is a short, one-line description of 
%       the problem. There is no need to restate the problem.
%    3. If a question has multiple parts, precede the answer to part x with the
%       command "\part{x}".
%    4. If a problem asks you to design an algorithm, use the commands
%       \algorithm, \correctness, \runtime to precede your discussion of the 
%       description of the algorithm, its correctness, and its running time, respectively.
%    5. You can include graphics by using the command \includegraphics{FILENAME}
%
\documentclass[11pt]{article}
\usepackage{mathtools}
\usepackage{graphicx}
\usepackage{subfigure}
\usepackage{amssymb}
\usepackage{multirow}
\usepackage{algorithm}
\usepackage{url}
\usepackage[noend]{algorithmic}
\usepackage[numbers,sort&compress]{natbib}
\usepackage{color}
\usepackage{colortbl}
\usepackage{multirow}
\usepackage{hhline}
\usepackage{textcomp}
\setlength{\arrayrulewidth}{1pt}

\hyphenation{op-tical net-works semi-conduc-tor}

\newtheorem{theorem}{\textbf{Theorem}}
\newtheorem{lemma}{\textbf{Lemma}}
\newtheorem{proposition}{\textbf{Proposition}}
\newtheorem{definition}{\textbf{Definition}}
\newtheorem{corollary}{\textbf{Corollary}}
\newtheorem{conjecture}{\textbf{Conjecture}}
\usepackage{amsmath,amssymb,amsthm}
\usepackage{graphicx}
\usepackage{subfigure}
\usepackage[margin=1in]{geometry}
\usepackage{fancyhdr}
\usepackage{algorithm}
\usepackage{url}
\usepackage[noend]{algorithmic}
\setlength{\parindent}{0pt}
\setlength{\parskip}{5pt plus 1pt}
\setlength{\headheight}{13.6pt}
\newcommand\question[2]{\vspace{.25in}\hrule\textbf{#1: #2}\vspace{.5em}\hrule\vspace{.10in}}
\renewcommand\part[1]{\vspace{.10in}\textbf{(#1)}}
%\newcommand\algorithm{\vspace{.10in}\textbf{Algorithm: }}
\newcommand\prof{\vspace{.10in}\textbf{Proof: }}
\newcommand\correctness{\vspace{.10in}\textbf{Correctness: }}
\newcommand\base {\vspace{.10in}\textbf{Base Case: }}
\newcommand\inductive {\vspace{.10in}\textbf{Inductive Step: }}
\newcommand\runtime{\vspace{.10in}\textbf{Running time: }}
\pagestyle{fancyplain}
\lhead{\textbf{\NAME}}
\lhead{\textbf{\NAME\ (\ID)}}
\chead{\textbf{\HWNUM}}
\rhead{CMPT 825 and CMPT 413, Fall 2017}
%\begin{document}\raggedright
\begin{document}
%Section A==============Change the values below to match your information==================
\newcommand\NAME{Turash and Luiz}  % your name
\newcommand\ID{tmosharr \& lperesde}     % your Temple AccessNet Account
\newcommand\HWNUM{}              % the homework number



%Section B==============Put your answers to the questions below here=======================

% no need to restate the problem --- the graders know which problem is which,
% but replacing "The First Problem" with a short phrase will help you remember
% which problem this is when you read over your homeworks to study.
%send to : HW3Subm.43vjts2a9ie04m7x@u.box.com
\question{A}{Theory} 

The algorithm used on this assignment was the \textbf{averaged perceptron} learning algorithm (as described on \textit{pg. 36, Sarkar 2011.}). The algorithm is a type of linear classifier that makes predictions based on a predictor function that combines a set of weights with the feature vector and outputs the averaged weight parameter vector. The details of our implementation is given on the section below.

\question{B}{Algorithm} 

The algorithm for the weight vector update is given below:		

\begin{algorithm}[htb]
	\renewcommand{\algorithmicrequire}{\textbf{Input:}\hspace{10.7pt}}
	\renewcommand\algorithmicensure {\textbf{Output:} }
	\caption{Phrasal Chunking}
	\label{alg:pbg}
		\begin{algorithmic}[1]
		\vspace{3pt}
		\REQUIRE
		Tagset file `$tagsetfile$`, Train data $train\_data$, Set of tags $tagset$ and \# of epochs of the algorithm $numepochs$\\
		\ENSURE Featured weight vector \\
		\STATE initialize $feat\_vector$ and $cumulative\_feat\_vec$ with zeroes
		\STATE initialize $epoch$ and $count$ with zero
		\WHILE{$epoch$ is less than $numepoch$}
		  \STATE initialize $mistakes$ and $correct$ with zero
		  \FOR{$sentence\_data$ in $train\_data$}
		    \STATE initialize $words$, $postags$ and $truetags$ as empty lists
		    \STATE initialize $label\_list$ as $sentence\_data[0]$ and $feat\_list$ as $sentence\_data[1]$	    	
		    \FOR{$label$ in $label\_list$}
		      \STATE \textbf{split} $label$ by \textit{spaces} and assign it to a triple $(word, postag, chunktag$ 
		      \STATE append $word$ to $words$ list, $postag$ to $postags$ list and $chunktag$ to $truetags$ list
		    \ENDFOR		    
		    \STATE initialize $tagset$ with $tagsetfile$ and $default\_tag$ as $tagset[0]$
		    \STATE initialize $argmaxtags$ as the result of $perc\_test$ passing $feat\_vec$, $label\_list$, $feat\_list$, $tagset$ and $default\_tag$ as params.
		    \STATE initialize $feat\_index$ and $i$ as zero
		    \FOR{$word$ in $words$}
		      \STATE get feats c for word with params $feat\_index$ and $feat\_list$
		      \STATE initialize $arg\_max$ with $argmaxtags[i]$ and $tru$ with $truetags[i]$
		      \IF{argamax equal tru}
		        \STATE increment $i$ and go back to next loop iteration		      
		      \ENDIF
		      \FOR{$f$ in $feats\_for\_this\_word$}
		        \STATE initialize $wrongkey$ with tuple $(f, argmax)$ and $rightkey$ with tuple $(f, tru)$
		        \STATE decrement value of $feat\_vec[wrongkey]$ and increment value of $feat\_vec[rightkey]$
		      \ENDFOR
		      \STATE increment $i$
		    \ENDFOR
		    \STATE set $i$ to zero
		    \FOR{$word$ in $words$}
		      \STATE initialize $arg\_max$ with $argmaxtags[i]$ and $tru$ with $truetags[i]$
		      \IF{argamax equal tru}
		        \STATE increment $correct$ and increment $i$. Then, go back to next loop iteration
		      \ELSE
		        \STATE increment $mistakes$		      
		      \ENDIF
		      \STATE initialize $argmaxprev$ and $truprev$ with \textit{"B:"}
		      \IF{$i$ equal zero}
		        \STATE concatenate $argmaxprev$ and $truprev$ with \textit{"B -1"}
		      \ELSE
		        \STATE concatenate $argmaxprev$ with $argmaxtags[i-1]$ and $truprev$ with $truetags[i-1]$
		      \ENDIF
		      \STATE initialize $wrongkey$ with tuple $(f, argmax)$ and $rightkey$ with tuple $(f, tru)$
		      \STATE decrement value of $feat\_vec[wrongkey]$ and increment value of $feat\_vec[rightkey]$
		      \STATE increment i
		    \ENDFOR
		    \FOR{$key$ in $feat\_vec.keys$}
		      \STATE set $cumulative\_feat\_vec[key]$ to the value of $cumulative\_feat\_vec[key]$ + $feat\_vec[key]$
		    \ENDFOR
		    \STATE increment count
		  \ENDFOR
		  \STATE increment $epoch$
		\ENDWHILE
		\FOR{$key$ in $cumulative\_feat\_vec$}
		  \STATE set $cumulative\_feat\_vec[key]$ = $cumulative\_feat\_vec[key]$/$count$
		\ENDFOR
		\STATE return $cumulative\_feat\_vec$
	\end{algorithmic}
\end{algorithm}

\end{document}
